\section{Future Work}

We identified a quite a few areas of future research that stems from this preliminary work. Next we describe those potential issues to be addressed.

\subsection{Size of re-key information}
Currently we generate a minimum amount of information that is required to re-key a peer and its contacts. But in this scheme the re-key information is of order n where n is the number of contacts of the peer. It would be interesting to evaluate the possibility of reducing the size of this public information while maintaining the same properties.

\subsection{Incentives to forward messages}
This scheme relies on the fact that the contacts belonging to the set $C^+$ (those who holds the latest $M_P$) will respond to an $update$ request $Q_P$ with correct a response $S_P$. It will be useful to evaluate the possibility of coming up with an incentive scheme for contacts in $C^+$ to respond to $Q_P$s. We need a mechanism where the responses can be evaluated for their correctness with the given context (time of the request etc.).

\subsection{Formal Proof of Security}
We have not provided a formal proof of security in this work and we plan to prove that, given a request message, an adversary with polynomially bounded resources will not be able to:
\begin{itemize}
\item Distinguish the contact who generated the request when compared with another request, and
\item Distinguish the valid response to the request given two responses (one valid and one not).
\item Infer the valid response message to the given response.
\end{itemize}

Furthermore we will prove that a polynomially bounded contact who was removed before a re-key operation will not be able to derive the new private key based on the public re-key information.

\subsection{Implementation of message routing}
The current implementation only covers the cryptographic primitives. It will be interesting to use these with a peer to peer network where the peers are connected only to their private contacts and evaluate the performance. It might be possible to be implemented as a plug-in to Freenet \cite{freenet}.

